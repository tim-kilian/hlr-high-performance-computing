\documentclass[11pt,a4paper]{article}
\usepackage[utf8]{inputenc}
\usepackage[T1]{fontenc}
\usepackage{a4wide}
\usepackage{graphicx}
\usepackage{epsfig}
\usepackage{amsmath, amssymb, amsfonts}
\usepackage{ngerman}
\usepackage{multirow}
\usepackage{hyperref}
\usepackage{float}
\usepackage{url}
\usepackage{csvsimple}
\usepackage{caption}

\usepackage{pgfplots}
\usepackage{csvsimple}
\begin{filecontents*}{data1.csv}
a,b
1, 54.91
2, 28.14
3, 19.24
4, 14.81
5, 12.35
6, 11.03
7, 10.02
8, 9.26
9, 8.55
10, 7.86
11, 7.11
12, 6.65

1, 54.98
2, 28.04
3, 19.29
4, 14.78
5, 12.52
6, 10.91
7, 10.03
8, 9.31
9, 8.72
10, 7.40
11, 7.05
12, 6.66

1, 54.93
2, 28.21
3, 19.25
4, 14.68
5, 12.40
6, 11.02
7, 10.02
8, 9.25
9, 8.55
10, 7.99
11, 7.17
12, 6.81
\end{filecontents*}

\begin{filecontents*}{data2.csv}
a,b,c,d
1, 54.91, 54.98, 54.93
2, 28.14, 28.04, 28.21
3, 19.24, 19.29, 19.25
4, 14.81, 14.78, 14.68
5, 12.35, 12.52, 12.40
6, 11.03, 10.91, 11.02
7, 10.02, 10.03, 10.02
8, 9.26, 9.31, 9.25
9, 8.55, 8.72, 8.55
10, 7.86, 7.40, 7.99
11, 7.11, 7.05, 7.17
12, 6.65, 6.66, 6.81
\end{filecontents*}

\begin{document}
    \Large\textbf{{{\noindent Übungsblatt 4 - Hochleistungsrechnen\\ }}}
    \textbf{{Leistungsanalyse\\}}
    Tim Kilian, Joscha Fregin, Stefan Knispel

    \section{Messung}

    \begin{tabular}{c|c|c|c}
        \bfseries Threads & \bfseries Time (seconds) & \bfseries Time (seconds) & \bfseries Time (seconds)
        \csvreader[head to column names]{data2.csv}{}
        {\\\hline\csvcoli&\csvcolii&\csvcoliii&\csvcoliv}
    \end{tabular}
    \\

    \begin{center}
	\begin{tikzpicture}
		\begin{axis}[ymin=0, xmin=1, xlabel={Threads}, ylabel={Time (seconds)}]
			\addplot table [x=a, y=b, col sep=comma] {data1.csv};
		\end{axis}
	\end{tikzpicture}

	\end{center}
	Je mehr Threads man hinzunimmt erhöht sich die Ausführungszeit des Programmes.
	Während jedochs anfangs die Leistungszunahme bei 50\% liegt,
	ist sie ab einer gewissen Anzahl an Threads kaum noch zu spüren.

\end{document}